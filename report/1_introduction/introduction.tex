
% this file is called up by thesis.tex
% content in this file will be fed into the main document

%: ----------------------- introduction file header -----------------------
\chapter{Introduction}

% the code below specifies where the figures are stored
\ifpdf
    \graphicspath{{1_introduction/figures/PNG/}{1_introduction/figures/PDF/}{1_introduction/figures/}}
\else
    \graphicspath{{1_introduction/figures/EPS/}{1_introduction/figures/}}
\fi

% ----------------------------------------------------------------------
%: ----------------------- introduction content ----------------------- 
% ----------------------------------------------------------------------

%: ----------------------- HELP: latex document organisation
% the commands below help you to subdivide and organise your thesis
%    \chapter{}       = level 1, top level
%    \section{}       = level 2
%    \subsection{}    = level 3
%    \subsubsection{} = level 4
% note that everything after the percentage sign is hidden from output

This study describes measuring the impact of the memory\footnote{\textit{CPU}: Central Processing Unit.} cache on data-sharing in multi-threaded\footnote{\textit{Thread}: a series of instructions that can be scheduled independently.} environments. It incorporates both theoretical and practical research. The theoretical investigation involved the creation of a taxonomy of inter-thread communication and of a simplified model that describes the inter-thread communication in a multi-core\footnote{\textit{Core}: an independent central processing units, any multi-core system consists of at least two cores.} environment. A number of experiments were designed. They characterise the cache performance, which is necessary for verification of the model. This thesis discusses the impact of cache and speculates about its positive influence on the speed of multi-threaded programmes. It continues the study that was conducted during last year \cite{Bazilinskyy2013}, where the negative effects of cache interference were discussed.

It is estimated that processor manufacturers will be using 5nm technology in 2019 \cite{Gruener2012,Iwai2009}, and some researchers claim that Moore's law \cite{Moore1998} will no longer be valid after this milestone has been reached \cite{Kish2002}. Others state that this is not the case, but that our civilization will meet considerable limitations in the area of manufacturing microprocessors soon \cite{lundstrom2003moore}. Therefore, investigating ways of improving efficiency of the current and future generations of central processing units (CPUs) is important. It is often assumed that the clock rate of a CPU is the main – and often the only – parameter that defines how fast and with what speed programmes are executed. However, a more accurate statement is that the speed of the CPU is not the only variable in the equation that dictates performance. Structure of cache, amount of disk memory, efficiency of compilers: all of these factors play their role. Additionally, all pieces of software have unique performance requirements. Optimised work of all of these components is crucial for achieving high performance \cite{Henning2000}.

Nowadays most software is executed on multi-core systems that provide load-balancing mechanisms for thread placement. Multi-core systems offer better performance due to parallelism \cite{Pusukuri}. Threads need to be \textit{scheduled} in a multi-core environment: decisions on which hardware resources need to be accessable by which threads must be taken. Load balancing of available resources is important for achieving satisfactory speed of execution of running programmes. Coscheduling programmes that involve multiple threads is complicated because of the complex architecture of multi-threaded systems \cite{Boyd-Wickizer2009,Peter2010}; it imposes numerous challenges and complications. These include: 1) excessive power consumption \cite{Tiwari1998}; 2) difficulties in achieving scalability \cite{Lin2006}; 3) avoiding deadlocks \cite{ousterhout1996threads}; 4) achieving portable and predictable performance. Multi-threaded programmes often utilise different patterns of cache usage. It is argued that optimisation of utilisation of cache in multi-core processors may be beneficial for optimising work of modern-day systems and reducing overhead\footnote{\textit{Overhead from the OS} is a decrease in speed of a programme caused by events that take place in the OS.} caused by these factors. Development of a cache-aware scheduler\footnote{\textit{Scheduler} controls access of threads or processes to processor time.} will be advantageous for the future of microprocessor design.

Also, such applications are commonly allocated on a single large heap shared by all threads and processes running on the OS. However, this approach may not always be considered as the best for a given application due to overhead caused by the mechanisms involved in inter-thread communication and their parallel execution. Moreover, programming multi-threaded code\footnote{\textit{Multi-threaded application} is a piece of software that incorporates more than one (main) thread.} often demands complex co-ordination of threads and can easily introduce subtle and difficult-to-find defects due to the interweaving of processing on data shared between threads, which may result in deployment of undependable software systems. It is argued that in certain situations running programmes on multiple cores must be avoided.

Numerous companies around the world, such as Openet\footnote{\url{http://www.openet.com}} and Intel\footnote{\url{http://www.intel.com}}, are investing large quantities of money and resources on optimising their hardware and software systems for more efficient use of the parallel programming paradigm. In the post-Moore's law era, the efficiency of hardware, and not software, will be of bigger importance. This particular interest in the IT industry gave motivation for this research. The project was initiated by Openet. It is a Dublin-based company that works in the area of telecommunications and their services are used to analyse and commercialize activity on the network. This thesis focuses on a basic case of a multi-threaded programme, which involves two threads: one sending thread and one receiving thread (described in \ref{modelsection}). A more realistic scenario where thousands or millions of threads are used can be evaluated based on the findings in this document.

\section{The Aim of the Thesis}
\label{final_aim}

The aim of this thesis is to investigate the impact of thread placement on the performance of communication between threads that reside on different cores/CPU chips\footnote{\textit{Chip}: an integrated circuit that contains the entire central processing unit (core)\cite{FreeOnlineDictionary2014}.} in applications run on systems that have various levels of multi-core inter-process communication. The thesis describes the specific case of data transfer from one thread to another via shared memory, and not data sharing in general. Such analysis is intended to help to determine how scheduling should take this into account.

Based on the results, the importance of thread placement\footnote{\textit{Thread placement} -- deciding which unit of computation a thread needs to be assigned to.} when scheduling decisions are made is discussed. The main research question RQ and four secondary research questions RQ1 -- RQ4 are asked in the study:

\begin{description}
  \item[RQ] Should a scheduler take into account where a receiving thread is executed?
  \item[RQ1] At what stage does allocation of threads that are engaged in intensive data-sharing on \textit{different cores of the same CPU chip} increase the speed of execution?
  \item[RQ2] At what stage does allocation of threads that are engaged in intensive data-sharing on \textit{different CPU chips} increase the speed of execution?
  \item[RQ3] Can the model describe the inter-thread communication with enough level of precision?
  \item[RQ4] Can the model be used to develop a cache-aware scheduler?
\end{description}

Six objectives were outlined to achieve the final aim, answer all research questions, and define the evaluation criteria:

\begin{enumerate}
  \item Develop a mathematical model of multi-core cache communication for both single- and multi-chip systems. Such model needs to be outlined because it will allow to predict to a certain degree of accuracy the impact of the decision made for scheduling the receiving thread for any CPU, for which the parameters can be obtained for and that matches the model. Without the model, the impact of cache on inter-thread communication would have to be analysed for each CPU on an individual basis. The behaviour of cache needs to be understood. For succeeding with this objective, a taxonomy of inter-thread communication needs to be created.
   \item Determine parameters for the systems used in the study. Collect data on latency of accessing different levels of cache and memory.
   \item Predict the impact of thread placement for different buffer lengths.
   \item Evaluate performance of the model through designing a series of application-level experiments. Working with such test cases where data needs to pass through different configurations of levels of cache and computer memory allows to compare the predictions of the model and data gathered in the real-life environments. Received data needs to be filtered\footnote{\textit{Filtered} data shows results with as little overhead from the OS as possible. The experiments were designed to explicitly measure overhead, and discount results that included it.} by detecting occurrences of interrupts and page faults: time-consuming events that are handled by the Linux kernel and cannot be avoided by real-world applications. Achieved results also help evaluate the taxonomy.
  \item Validate the model against the data gathered in a real-world setting.
  \item Use the model and experiment results to determine the impact of placement on performance.
%   \item Compare the received results with information provided in official specifications of used in experiments hardware, is such documentation is available.
\end{enumerate}

Answers to these questions are given in chapter \ref{discussionChapter}. The mathematical model that described inter-thread communication is presented in chapter \ref{chapterModel}. The outcome of this study provides a basis for thread scheduling to take cache performance into account. In addition, such results may be used for development of a cache-aware scheduler.

\section{Dissertation Structure}
The remainder of this dissertation is structured as follows: chapter 2 surveys the related content applicable to this thesis, it gives a brief description of multi-core systems and the cache. Chapter 3 introduces a taxonomy and a mathematical model of inter-thread communication. Chapter 4 gives a description of the experimental environment and experiments used in the practical section of the study. Then, chapter 5 outlines the process of conducting experiments. It also includes a brief overview of the hardware used in the project described in this thesis. Chapter 6 talks about achieved results. Chapter 7 gives the evaluation of achieved results. Lastly, chapter 8 sums up what was accomplished and sketches possible directions for future work.

% \section{put section name here} % section headings are printed smaller than chapter names
% % intro
% Write your text without any further commands, like this:.... Any organised system requires energy, be it a machine of some kind or a live organism. Energy is needed to win the uphill battle against entropy and pull together lifeless molecules to be able to do something in this world, like complete a PhD.



% \subsection{Name your subsection} % subsection headings are again smaller than section names
% % lead
% Different organised systems have different energy currencies. The machines that enable us to do science like sizzling electricity but at a controlled voltage. Earth's living beings are no different, except that they have developed another preference. They thrive on various chemicals. 

% % dextran, starch, glycogen
% Most organisms use polymers of glucose units for energy storage and differ only slightly in the way they link together monomers to sometimes gigantic macromolecules. Dextran of bacteria is made from long chains of $\alpha$-1,6-linked glucose units. 

% %: ----------------------- HELP: special characters
% % above you can see how special characters are coded; e.g. $\alpha$
% % below are the most frequently used codes:
% %$\alpha$  $\beta$  $\gamma$  $\delta$

% %$^{chars to be superscripted}$  OR $^x$ (for a single character)
% %$_{chars to be suberscripted}$  OR $_x$

% %>  $>$  greater,  <  $<$  less
% %≥  $\ge$  greater than or equal, ≤  $\ge$  lesser than or equal
% %~  $\sim$  similar to

% %$^{\circ}$C   ° as in degree C
% %±  \pm     plus/minus sign

% %$\AA$     produces  Å (Angstrom)




% % dextran, starch, glycogen continued
% Starch of plants and glycogen of animals consists of $\alpha$-1,4-glycosidic glucose polymers \cite{lastname07}. See figure \ref{largepotato} for a comparison of glucose polymer structure and chemistry. 

% Two references can be placed separated by a comma \cite{lastname07,name06}.

% %: ----------------------- HELP: references
% % References can be links to figures, tables, sections, or references.
% % For figures, tables, and text you define the target of the link with \label{XYZ}. Then you call cross-link with the command \ref{XYZ}, as above
% % Citations are bound in a very similar way with \cite{XYZ}. You store your references in a BibTex file with a programme like BibDesk.





% \figuremacro{largepotato}{A common glucose polymers}{The figure shows starch granules in potato cells, taken from \href{http://molecularexpressions.com/micro/gallery/burgersnfries/burgersnfries4.html}{Molecular Expressions}.}

% %: ----------------------- HELP: adding figures with macros
% % This template provides a very convenient way to add figures with minimal code.
% % \figuremacro{1}{2}{3}{4} calls up a series of commands formating your image.
% % 1 = name of the file without extension; PNG, JPEG is ok; GIF doesn't work
% % 2 = title of the figure AND the name of the label for cross-linking
% % 3 = caption text for the figure

% %: ----------------------- HELP: www links
% % You can also see above how, www links are placed
% % \href{http://www.something.net}{link text}

% \figuremacroW{largepotato}{Title}{Caption}{0.8}
% % variation of the above macro with a width setting
% % \figuremacroW{1}{2}{3}{4}
% % 1-3 as above
% % 4 = size relative to text width which is 1; use this to reduce figures




% Insulin stimulates the following processes:

% \begin{itemize}
% \item muscle and fat cells remove glucose from the blood,
% \item cells breakdown glucose via glycolysis and the citrate cycle, storing its energy in the form of ATP,
% \item liver and muscle store glucose as glycogen as a short-term energy reserve,
% \item adipose tissue stores glucose as fat for long-term energy reserve, and
% \item cells use glucose for protein synthesis.
% \end{itemize}

% %: ----------------------- HELP: lists
% % This is how you generate lists in LaTeX.
% % If you replace {itemize} by {enumerate} you get a numbered list.


 


% %: ----------------------- HELP: tables
% % Directly coding tables in latex is tiresome. See below.
% % I would recommend using a converter macro that allows you to make the table in Excel and convert them into latex code which you can then paste into your doc.
% % This is the link: http://www.softpedia.com/get/Office-tools/Other-Office-Tools/Excel2Latex.shtml
% % It's a Excel template file containing a macro for the conversion.

% \begin{table}[htdp]
% \centering
% \begin{tabular}{ccc} % ccc means 3 columns, all centered; alternatives are l, r

% {\bf Gene} & {\bf GeneID} & {\bf Length} \\ 
% % & denotes the end of a cell/column, \\ changes to next table row
% \hline % draws a line under the column headers

% human latexin & 1234 & 14.9 kbps \\
% mouse latexin & 2345 & 10.1 kbps \\
% rat latexin   & 3456 & 9.6 kbps \\
% % Watch out. Every line must have 3 columns = 2x &. 
% % Otherwise you will get an error.

% \end{tabular}
% \caption[title of table]{\textbf{title of table} - Overview of latexin genes.}
% % You only need to write the title twice if you don't want it to appear in bold in the list of tables.
% \label{latexin_genes} % label for cross-links with \ref{latexin_genes}
% \end{table}



% There you go. You already know the most important things.


% ----------------------------------------------------------------------



